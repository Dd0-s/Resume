\documentclass[a4paper,12pt]{article}
\usepackage[T2A]{fontenc}
\usepackage[utf8]{inputenc}	
\usepackage[english,russian]{babel}	
\usepackage{amsmath,amsfonts,amssymb,amsthm,mathtools} 
\usepackage[left=16mm,  top=15mm,  right=16mm,  bottom=20mm, nohead, nofoot]{geometry}
\usepackage{graphicx}
\graphicspath{{pictures/}}
\DeclareGraphicsExtensions{.pdf,.png,.jpg}
\usepackage{wasysym}
\usepackage{tikz}
\usepackage[unicode, pdftex]{hyperref}

\begin{document} 

\begin{center}
\Huge{\textbf{Резюме}}
\end{center}

\section*{Личные данные}

    \begin{enumerate}

        \item[$\bullet$] ФИО: Богданов Александр Иванович

        \item[$\bullet$] Ссылка на GitHub: \url{https://github.com/Dd0-s} 

        \item[$\bullet$] Почта: bogdanov.ai@phystech.edu

        \item[$\bullet$] Телеграмм: @d0dos

        \item[$\bullet$] Телефон: 89052111518

    \end{enumerate}

\section*{Образование}

    \begin{enumerate}

        \item[$\bullet$] Школа: закончил ФМЛ №239

        Средний балл: 4.71

        \item[$\bullet$] Вуз: студент 4 курса МФТИ ФПМИ ПМФ

        Средний балл: 8.63 (Получал Абрамовскую стипендию)

        \item[$\bullet$] Базовая кафедра: Интеллектуальный анализ данных (ИАД)

        \item[$\bullet$] ШАД: студент 1 курса

    \end{enumerate}

\section*{Навыки}

    \begin{enumerate}

        \item[$\bullet$] Языки программирования

            \begin{enumerate}

                \item[--] С/С++: МФТИ, ШАД (часть 1), Computer Science Center (часть 1, часть 2)

                \item[--] Python: ИАД, Bioinformatics Institute (часть 1, часть 2)

                \item[--] GO: ШАД
                
                \item[--] Прочее: Git, SQL, assembler, latex

            \end{enumerate}

        \item[$\bullet$] Машинное обучение

            \begin{enumerate}

                \item[--] Машинное обучение: МФТИ (часть 1, часть 2), ШАД (часть 1, часть 2)

                \item[--] Обучение с подкреплением: МФТИ, ШАД

                \item[--] Временные ряды: ИАД (часть 1, часть 2)

                \item[--] Глубокое обучение: ИАД

                \item[--] Рекомендательные системы: ИАД

                \item[--] Байесовский выбор моделей: ИАД (часть 1, часть 2)

            \end{enumerate}

        \item[$\bullet$] Прочие знания

            \begin{enumerate}

                \item[--] Методы оптимизация: МФТИ (часть 1, часть 2)

                \item[--] Алгоритмы и структуры данных: ШАД (часть 1), МФТИ, Computer Science Center

                \item[--] Теория вероятностей, случайные процессы, математическая статистика: МФТИ

            \end{enumerate}

    \end{enumerate}

\section*{Работа над статьями}

    \begin{enumerate}

        \item[$\bullet$] Методы оптимизации: <<New Aspects of Black Box Conditional Gradient: Variance Reduction and One Point Feedback>>

        \item[$\bullet$] Machine Learning: <<Классификация траекторий физических систем с помощью лагранжевых нейронных сетей>> \url{https://github.com/intsystems/2023-Project-114}

    \end{enumerate}

\section*{Работа}

    \begin{enumerate}

        \item[$\bullet$]  Работал (2019г - 2020г) в школе ассистентом на кружке <<Олимпиадная физика>>

        \item[$\bullet$] Работал (2021г) ассистентом по математике на <<Летней олимпиадной школе>>

        \item[$\bullet$] Работал (2022г) ассистентом по предмету <<Теория и реализация языков программирования>>

        \item[$\bullet$] Работал (2023г) ассистентом по предмету <<Алгоритмы и модели вычислений>>

        \item[$\bullet$] Работал (2023г) ассистентом по предмету <<Теория и алгоритмы моделей вычислений>>
        
        \item[$\bullet$] Работаю (2023г - н.в.) ассистентом по предмету <<Методы Оптимизации>>
        
        \item[$\bullet$] Работаю (2023г - н.в.) в <<Лаборатории математических методов оптимизации ФПМИ>> - пишу статью и там же буду писать диплом
        
    \end{enumerate}

\end{document}